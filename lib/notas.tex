Firebase apps:

Platform  Firebase App Id
web       1:426828972528:web:9f72cf49a221432a1cdd35
android   1:426828972528:android:c1ac8d84290466e51cdd35
ios       1:426828972528:ios:f27f9670cbc551c01cdd35
macos     1:426828972528:ios:44520e19324a2d381cdd35


1. For creating the keystore file i.e., debug.keystore 

keytool -genkey -v -keystore debug.keystore -alias androiddebugkey -storepass android -keypass android -keyalg RSA -keysize 2048 -validity 10000

2. For viewing the SHA-1 key from the keystore file that we created 

keytool -list -v -keystore "C:\Users\HYSE02\.android\debug.keystore" -alias androiddebugkey -storepass android -keypass android


SHA1: 67:19:57:F7:51:E6:30:F3:40:2F:4D:C5:65:9F:76:C5:81:8B:04:6B
SHA256: 43:70:4F:9C:1A:86:1A:19:E9:55:A4:42:A6:9A:02:58:1D:AA:5E:1A:71:DE:E6:B1:02:4E:15:12:17:F9:3C:A5













replace GT with greater than symbol
replace LT with less than symbol
youtube description don't like angle brackets

if you have not previous paired your phone with pico-w
then turn on the pico-w then on your phone settings
disconnect current wifi and connect to pico-w
ssid is 'MYPICO' password is "1234567890"
then turn off the pico-w

1) run the flutter program in your phone
2) press listen button
3) start in thonny micropython program below (very important)
4) press connect button
5) press refresh to get the lm75 temperature from pico-w

note: do not start pico-w micropython program before your flutter app

wiring pico-w to lm75 from electricdollarstore.com
lm75    pico w
scl  --  gpio5
sda  --  gpio4
gnd  --  any gnd
3.3v --  pin 36

pico w micropython program
import network
import socket
from machine import Pin, I2C
from time import sleep

i2c=I2C(0)
ap=network.WLAN(network.AP_IF)
ap.config(essid="MYPICO", password="1234567890")
ap.active(True)

while ap.active() == False:
    pass

s=socket.socket()
s.bind(('0.0.0.0', 8000))
s.listen(1)
s2=socket.socket()
s2.connect(('192.168.4.16', 9000))
while True:
    s1, remote=s.accept()
    while True:
        data=s1.readline()
        buf=i2c.readfrom(0x48,2)
        s2.write(buf)

        
flutter android program
import 'package:flutter/material.dart';
import 'package:get/get.dart';
import 'dart:async';
import 'dart:io';
import 'dart:convert';

void main() =_GT_ runApp(GetMaterialApp(home: Home()));
class Home extends StatelessWidget {
late Socket socket;
RxString picoData = ' '.obs;
double value = 0;

void send(String c){socket.add(utf8.encode(c));}
  
void connect() async {
socket = await Socket.connect('192.168.4.1', 8000);} 

// stackoverflow https://stackoverflow.com/questions/7...
void handleClient(Socket client) {
    print('server incoming connection from ${client.remoteAddress.address}:${client.remotePort}');
    client.listen((data) {
      value = ((data[0] LT_LT 3) | (data[1] GT__GT 5)) * 0.125;
      picoData.value = value.toString();
    }, onDone: () {
      print("server done");
    });
    client.close();}

  void listen() {
    ServerSocket.bind('0.0.0.0', 9000).then((ServerSocket server) {
      server.listen(handleClient);});}

@override
Widget build(context) {
return Scaffold(
appBar: AppBar(title: Text('2 way socket test with pico w'),),
body:
  Padding(padding: EdgeInsets.all(20.0),
    child:Column(children: [
    
      Text('Press listen button then plugin the pico-w not before, then press connect button'), 
      SizedBox(height:15),
      ElevatedButton(child: Text('listen'), onPressed:(){ listen(); }),
      SizedBox(height:15),
      ElevatedButton(child: Text('connect'), onPressed:(){ connect(); }),
      SizedBox(height:20),
      
      Row(children:[
        SizedBox(width:20),
        Text('LM75 temp from Pico W: ', style: TextStyle(fontSize:20, fontWeight:FontWeight.bold)),
        Obx(() =_GT_ Text('${picoData}', style: TextStyle(fontSize:30, fontWeight:FontWeight.bold))),
      ]),
      
      SizedBox(height:20),
      ElevatedButton(child: Text('refresh'), onPressed:(){send('refresh\n');})
      
])));}}






